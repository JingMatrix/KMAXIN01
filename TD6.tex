\documentclass[solutions]{exercices}

\usepackage{enumitem}
\usepackage{hyperref}

\begin{document}

\newcommand{\e}{\mathrm{e}}

\makeheader{2025}{2026}
\tdtitle{6}{Théorèmes de convergence}

\begin{exercice}[\di] \emph{La série harmonique.}
	Soit $H_n =\sum\limits_{k=1}^n\dfrac1k= 1 + \dfrac12 + \cdots + \dfrac1n$.
	\begin{enumerate}
		\item En utilisant une intégrale, montrer que pour tout entier $k>0$ : 
			\[
				\dfrac1{k + 1} \leq \ln (k + 1)-\ln (k) \leq \dfrac1k.
			\]
		\item En déduire que $\ln (n + 1) \leq H_n \leq \ln (n) + 1$.
		\item Déterminer la limite de $H_n$.
		\item Montrer que $u_n = H_n-\ln (n)$ est décroissante et positive.
		\item Conclure.
	\end{enumerate}
\end{exercice}

\begin{solution}
	\begin{enumerate}
		\item Soit $k > 0$ un entier. La fonction $t \mapsto \frac{1}{t}$ est continue et décroissante sur l'intervalle $[k, k+1]$.
		      Par conséquent, pour tout $t \in [k, k+1]$, on a :
		      \[ \frac{1}{k+1} \le \frac{1}{t} \le \frac{1}{k} \]
		      En intégrant cette inégalité sur l'intervalle $[k, k+1]$ (par croissance de l'intégrale), on obtient :
		      \[ \int_k^{k+1} \frac{1}{k+1} \,dt \le \int_k^{k+1} \frac{1}{t} \,dt \le \int_k^{k+1} \frac{1}{k} \,dt \]
		      \[ \frac{1}{k+1} [t]_k^{k+1} \le [\ln(t)]_k^{k+1} \le \frac{1}{k} [t]_k^{k+1} \]
		      \[ \frac{1}{k+1} (k+1-k) \le \ln(k+1) - \ln(k) \le \frac{1}{k} (k+1-k) \]
		      Ce qui donne bien : $\dfrac1{k + 1} \leq \ln (k + 1)-\ln (k) \leq \dfrac1k$.

		\item \textbf{Majoration de $H_n$ :} On utilise l'inégalité de droite $\frac{1}{k} \ge \ln(k+1)-\ln(k)$. En sommant pour $k$ de 1 à $n$ :
		      \[ H_n = \sum_{k=1}^n \frac{1}{k} \ge \sum_{k=1}^n (\ln(k+1)-\ln(k)) \]
		      La somme de droite est une somme télescopique :
		      \[ (\ln(2)-\ln(1)) + (\ln(3)-\ln(2)) + \dots + (\ln(n+1)-\ln(n)) = \ln(n+1) - \ln(1) = \ln(n+1) \]
		      On a donc montré $H_n \ge \ln(n+1)$. (Note: il y a une inversion dans l'énoncé, ceci est la minoration).

		      \textbf{Minoration de $H_n$ (Majoration dans l'énoncé) :} On utilise l'inégalité de gauche $\frac{1}{k} \le \ln(k)-\ln(k-1)$ pour $k \ge 2$.
		      \[ H_n = 1 + \sum_{k=2}^n \frac{1}{k} \le 1 + \sum_{k=2}^n (\ln(k)-\ln(k-1)) \]
		      La somme est télescopique : $(\ln(2)-\ln(1)) + \dots + (\ln(n)-\ln(n-1)) = \ln(n)-\ln(1) = \ln(n)$.
		      On a donc montré $H_n \le 1 + \ln(n)$.

		\item D'après la question 2, on a l'encadrement $\ln(n+1) \le H_n$.
		      Comme $\lim_{n\to\infty} \ln(n+1) = +\infty$, par le théorème de comparaison, on conclut que $\lim_{n\to\infty} H_n = +\infty$.

		\item Soit $u_n = H_n - \ln(n)$. Pour étudier sa monotonie, on calcule $u_{n+1}-u_n$ :
		      \[ u_{n+1}-u_n = (H_{n+1}-\ln(n+1)) - (H_n-\ln(n)) = (H_{n+1}-H_n) - (\ln(n+1)-\ln(n)) \]
		      \[ u_{n+1}-u_n = \frac{1}{n+1} - (\ln(n+1)-\ln(n)) \]
		      D'après la question 1 (avec $k=n$), on sait que $\ln(n+1)-\ln(n) \ge \frac{1}{n+1}$.
		      Donc, $u_{n+1}-u_n \le 0$. La suite $(u_n)$ est décroissante.

		      De plus, d'après la question 2, $H_n \ge \ln(n+1)$. Comme $\ln(n+1) > \ln(n)$, on a $H_n > \ln(n)$, donc $u_n = H_n-\ln(n) > 0$. La suite est positive.

		\item La suite $(u_n)$ est décroissante et minorée (par 0). D'après le théorème de la convergence monotone, elle converge vers une limite finie, notée $\gamma$. Cette limite est la constante d'Euler-Mascheroni.
	\end{enumerate}
\end{solution}

\begin{exercice}[\di] \emph{Le nombre $\e$ est irrationnel}
	On considère les suites $(u_n)_{n\in\mathbb{N}^*}$ et $(v_n)_{n\in\mathbb{N}^*}$ suivantes :
	\[u_n=\sum_{k=0}^{n}\dfrac{1}{k!} \quad \text{et} \quad v_n=u_n+\frac1{n\cdot n!} \]
	\begin{enumerate}
		\item Démontrer que les deux suites $(u_n)$ et $(v_n)$ sont adjacentes. On admet que leur limite commune est le nombre $\e$.
		\item Justifier que pour tout $n\in\mathbb{N}^*$, on a $u_n<\e<v_n$.
		\item Soit $n\in\mathbb{N}^*$. Démontrer qu'il existe un entier $A_n$ tel que $A_n<n\cdot n!\e<A_n+1$.
		\item Montrer que si $\e\in\mathbb{Q}$ alors il existe un entier $n$ pour lequel $n\cdot n!\e\in\mathbb{N}$. En déduire que $\e\notin\mathbb{Q}$.
	\end{enumerate}
\end{exercice}

\begin{solution}
	\begin{enumerate}
		\item Pour montrer que $(u_n)$ et $(v_n)$ sont adjacentes, on vérifie trois points :
		      \begin{itemize}
			      \item \textbf{Monotonie de $(u_n)$ :} $u_{n+1}-u_n = \sum_{k=0}^{n+1} \frac{1}{k!} - \sum_{k=0}^{n} \frac{1}{k!} = \frac{1}{(n+1)!} > 0$. Donc $(u_n)$ est strictement croissante.
			      \item \textbf{Monotonie de $(v_n)$ :}
			            $v_{n+1}-v_n = (u_{n+1}-u_n) + \frac{1}{(n+1)(n+1)!} - \frac{1}{n \cdot n!} = \frac{1}{(n+1)!} + \frac{1}{(n+1)(n+1)!} - \frac{1}{n \cdot n!}$.
			            En mettant au dénominateur commun $n(n+1)(n+1)!$ :
			            $v_{n+1}-v_n = \frac{n(n+1) + n - (n+1)^2}{n(n+1)(n+1)!} = \frac{n^2+n+n - (n^2+2n+1)}{n(n+1)(n+1)!} = \frac{-1}{n(n+1)(n+1)!} < 0$.
			            Donc $(v_n)$ est strictement décroissante.
			      \item \textbf{Limite de la différence :} $\lim_{n\to\infty} (v_n - u_n) = \lim_{n\to\infty} \frac{1}{n \cdot n!} = 0$.
		      \end{itemize}
		      Les trois conditions sont vérifiées, les suites sont adjacentes.
		\item Comme $(u_n)$ est strictement croissante et converge vers $e$, on a $u_n < e$ pour tout $n$.
		      Comme $(v_n)$ est strictement décroissante et converge vers $e$, on a $v_n > e$ pour tout $n$.
		      Donc, $u_n < e < v_n$.
		\item On part de l'inégalité $u_n < e < v_n$ et on remplace $v_n$ par sa définition :
		      \[ \sum_{k=0}^{n} \frac{1}{k!} < e < \sum_{k=0}^{n} \frac{1}{k!} + \frac{1}{n \cdot n!} \]
		      On multiplie l'ensemble de l'inégalité par $n \cdot n! > 0$ :
		      \[ n \cdot n! \sum_{k=0}^{n} \frac{1}{k!} < n \cdot n! e < n \cdot n! \sum_{k=0}^{n} \frac{1}{k!} + n \cdot n! \frac{1}{n \cdot n!} \]
		      \[ \sum_{k=0}^{n} \frac{n \cdot n!}{k!} < n \cdot n! e < \sum_{k=0}^{n} \frac{n \cdot n!}{k!} + 1 \]
		      Posons $A_n = \sum_{k=0}^{n} \frac{n \cdot n!}{k!}$. Pour $k \le n$, $k!$ divise $n!$, donc $\frac{n!}{k!}$ est un entier. Par conséquent, $A_n$ est un entier. On a bien trouvé un entier $A_n$ tel que $A_n < n \cdot n! e < A_n + 1$.
		\item Raisonnons par l'absurde. Supposons que $e$ est rationnel, c'est-à-dire $e = \frac{p}{q}$ avec $p, q \in \mathbb{N}^*$.
		      Considérons le nombre $X = q \cdot q! e = q \cdot q! \frac{p}{q} = (q-1)! p$. C'est un produit d'entiers, donc $X$ est un entier.
		      Or, d'après la question 3, en choisissant $n=q$, on sait qu'il existe un entier $A_q$ tel que :
		      \[ A_q < q \cdot q! e < A_q + 1 \]
		      \[ A_q < X < A_q + 1 \]
		      Ceci est une contradiction, car $X$ est un entier et il ne peut pas être strictement compris entre deux entiers consécutifs.
		      L'hypothèse "$e$ est rationnel" est donc fausse. On conclut que $e \notin \mathbb{Q}$.
	\end{enumerate}
\end{solution}

\begin{exercice}[\di]
	Soit la suite $(u_n)$ définie par
	\[
		u_n=\sum_{k=1}^{n}\frac{(-1)^{k+1}}{k}.
	\]
	On pose $v_n=u_{2n}$ et $w_n=u_{2n+1}$.
	\begin{enumerate}
		\item Calculer $v_1,v_2,v_3$ et $w_1,w_2,w_3$.
		\item Démontrer que les suites $(v_n)$ et $(w_n)$ sont adjacentes.
		\item En déduire la convergence de la suite $(u_n)$.
	\end{enumerate}
\end{exercice}

\begin{solution}
	\begin{enumerate}
		\item Calculs :
		      $v_1 = u_2 = 1 - \frac{1}{2} = \frac{1}{2}$.
		      $v_2 = u_4 = 1 - \frac{1}{2} + \frac{1}{3} - \frac{1}{4} = \frac{6+2-3}{12} = \frac{5}{12}$.
		      $v_3 = u_6 = \frac{5}{12} + \frac{1}{5} - \frac{1}{6} = \frac{25+12-10}{60} = \frac{27}{60} = \frac{9}{20}$.
		      $w_1 = u_3 = 1 - \frac{1}{2} + \frac{1}{3} = \frac{5}{6}$.
		      $w_2 = u_5 = \frac{5}{6} - \frac{1}{4} + \frac{1}{5} = \frac{50-15+12}{60} = \frac{47}{60}$.
		      $w_3 = u_7 = \frac{47}{60} - \frac{1}{6} + \frac{1}{7} = \frac{329-70+60}{420} = \frac{319}{420}$.
		\item Montrons que $(v_n)$ et $(w_n)$ sont adjacentes.
		      \begin{itemize}
			      \item \textbf{Monotonie de $(v_n)$ :} $v_{n+1}-v_n = u_{2n+2}-u_{2n} = \frac{(-1)^{2n+2}}{2n+1} + \frac{(-1)^{2n+3}}{2n+2} = \frac{1}{2n+1} - \frac{1}{2n+2} = \frac{1}{(2n+1)(2n+2)} > 0$. Donc $(v_n)$ est croissante.
			      \item \textbf{Monotonie de $(w_n)$ :} $w_{n+1}-w_n = u_{2n+3}-u_{2n+1} = \frac{(-1)^{2n+3}}{2n+2} + \frac{(-1)^{2n+4}}{2n+3} = -\frac{1}{2n+2} + \frac{1}{2n+3} = \frac{-1}{(2n+2)(2n+3)} < 0$. Donc $(w_n)$ est décroissante.
			      \item \textbf{Limite de la différence :} $w_n - v_n = u_{2n+1}-u_{2n} = \frac{(-1)^{2n+2}}{2n+1} = \frac{1}{2n+1}$.
			            $\lim_{n\to\infty} (w_n - v_n) = \lim_{n\to\infty} \frac{1}{2n+1} = 0$.
		      \end{itemize}
		      Les trois conditions sont vérifiées, les suites sont adjacentes.
		\item Puisque $(v_n)$ et $(w_n)$ sont adjacentes, elles convergent vers la même limite $\ell$.
		      La suite $(v_n)$ est la suite des termes de rang pair de $(u_n)$. La suite $(w_n)$ est la suite des termes de rang impair de $(u_n)$.
		      Comme les suites des termes de rangs pair et impair convergent vers la même limite $\ell$, on peut conclure que la suite $(u_n)$ converge vers $\ell$. (On admettra que la limite est $\ln(2)$).
	\end{enumerate}
\end{solution}

\begin{exercice}[\di]
	Soient $a$ et $b$ deux réels tels que $0 < a < b$ et les deux suites $u$ et $v$ définies par :
	\[ u_0 = a, \quad u_{n+1} = \sqrt{u_n v_n} \quad \text{et} \quad v_0 = b, \quad v_{n+1} = \frac{u_n + v_n}{2}. \]
	\begin{enumerate}
		\item Montrer que, pour tout $n \in \mathbb N$, $u_n$ et $v_n$ sont bien définis et $0<u_n < v_n$.
		\item Montrer que $(u_n)$ est strictement croissante et $(v_n)$ strictement décroissante.
		\item En déduire que $u$ et $v$ sont convergentes et ont même limite.
	\end{enumerate}
\end{exercice}

\begin{solution}
	\begin{enumerate}
		\item On procède par récurrence. Soit $P_n$ la proposition : "$u_n$ et $v_n$ sont définis et $0 < u_n < v_n$".
		      \textbf{Initialisation :} Pour $n=0$, on a $0 < a < b$, donc $0 < u_0 < v_0$. $P_0$ est vraie.
		      \textbf{Hérédité :} Supposons $P_n$ vraie pour un certain $n \ge 0$. On a $0 < u_n < v_n$.
		      Alors $u_n$ et $v_n$ sont strictement positifs, donc $u_{n+1}=\sqrt{u_n v_n}$ et $v_{n+1}=\frac{u_n+v_n}{2}$ sont bien définis et strictement positifs.
		      On compare $u_{n+1}$ et $v_{n+1}$. On sait que l'inégalité arithmético-géométrique stipule que pour des réels positifs, la moyenne géométrique est inférieure à la moyenne arithmétique.
		      \[ \sqrt{u_n v_n} \le \frac{u_n+v_n}{2} \]
		      L'égalité a lieu si et seulement si $u_n=v_n$. Comme on a supposé $u_n < v_n$, l'inégalité est stricte : $u_{n+1} < v_{n+1}$.
		      On a donc bien $0 < u_{n+1} < v_{n+1}$. $P_{n+1}$ est vraie.
		      Par récurrence, $P_n$ est vraie for tout $n \in \mathbb{N}$.
		\item \textbf{Monotonie de $(u_n)$ :} On compare $u_{n+1}$ et $u_n$.
		      $u_{n+1} = \sqrt{u_n v_n}$. Comme $u_n < v_n$, on a $u_n^2 < u_n v_n$. En prenant la racine (croissante), $\sqrt{u_n^2} < \sqrt{u_n v_n}$, soit $u_n < u_{n+1}$. La suite $(u_n)$ est strictement croissante.

		      \textbf{Monotonie de $(v_n)$ :} On compare $v_{n+1}$ et $v_n$.
		      $v_{n+1} = \frac{u_n+v_n}{2}$. Comme $u_n < v_n$, on a $u_n+v_n < v_n+v_n = 2v_n$. Donc $\frac{u_n+v_n}{2} < v_n$, soit $v_{n+1} < v_n$. La suite $(v_n)$ est strictement décroissante.
		\item La suite $(u_n)$ est croissante et majorée par $v_0$ (car $u_n < v_n < v_{n-1} < \dots < v_0$). D'après le théorème de la convergence monotone, $(u_n)$ converge vers une limite $\ell_u$.
		      La suite $(v_n)$ est décroissante et minorée par $u_0$. Elle converge donc vers une limite $\ell_v$.
		      On a la relation $v_{n+1} = \frac{u_n+v_n}{2}$. En passant à la limite quand $n \to \infty$ :
		      \[ \lim v_{n+1} = \lim \frac{u_n+v_n}{2} \]
		      \[ \ell_v = \frac{\ell_u + \ell_v}{2} \]
		      \[ 2\ell_v = \ell_u + \ell_v \implies \ell_v = \ell_u \]
		      Les deux suites convergent vers la même limite.
	\end{enumerate}
\end{solution}

\begin{exercice}
	Soit $(u_{n})$ une suite numérique. On suppose que les suites extraites $(u_{2n})$, $(u_{2n + 1})$, et $(u_{3n})$ convergent et on note leurs limites respectives $l_1$, $l_2$ et $l_3$.
	\begin{enumerate}
		\item Montrer que la suite $(u_{6n})$ est une suite extraite de $(u_{2n})$ et de $(u_{3n})$. En déduire que $l_1=l_3$.
		\item Montrer que la suite $(u_{6n+3})$ est une suite extraite de $(u_{2n+1})$ et de $(u_{3n})$. En déduire que $l_2=l_3$.
		\item En déduire que $(u_n)$ converge.
	\end{enumerate}
\end{exercice}

\begin{solution}
	\begin{enumerate}
		\item La suite $(u_{6n})$ est la suite des termes d'indices $0, 6, 12, 18, \dots$.
		      Elle est extraite de $(u_{2n})$ (indices $0,2,4,\dots$) car $6n = 2 \times (3n)$, c'est la sous-suite des termes de rang $3n$ de $(u_{2n})$.
		      Elle est extraite de $(u_{3n})$ (indices $0,3,6,\dots$) car $6n = 3 \times (2n)$, c'est la sous-suite des termes de rang $2n$ de $(u_{3n})$.

		      Comme $(u_{2n}) \to l_1$, toute suite extraite de $(u_{2n})$ converge vers $l_1$. Donc $\lim_{n\to\infty} u_{6n} = l_1$.
		      Comme $(u_{3n}) \to l_3$, toute suite extraite de $(u_{3n})$ converge vers $l_3$. Donc $\lim_{n\to\infty} u_{6n} = l_3$.
		      Par unicité de la limite, on conclut que $l_1=l_3$.
		\item La suite $(u_{6n+3})$ est la suite des termes d'indices $3, 9, 15, 21, \dots$.
		      L'indice $6n+3 = 3(2n+1)$ est un multiple impair de 3. C'est la sous-suite des termes de rang $2n+1$ de $(u_{3n})$. Comme $(u_{3n}) \to l_3$, on a $\lim_{n\to\infty} u_{6n+3} = l_3$.
		      L'indice $6n+3 = 2(3n+1)+1$ est un indice impair. C'est la sous-suite des termes de rang $3n+1$ de $(u_{2n+1})$. Comme $(u_{2n+1}) \to l_2$, on a $\lim_{n\to\infty} u_{6n+3} = l_2$.
		      Par unicité de la limite, on conclut que $l_2=l_3$.
		\item Des questions 1 et 2, on a $l_1 = l_3$ et $l_2 = l_3$, donc $l_1 = l_2$.
		      La suite des termes de rang pair $(u_{2n})$ et la suite des termes de rang impair $(u_{2n+1})$ convergent vers la même limite.
		      Ceci est une condition suffisante pour affirmer que la suite $(u_n)$ converge vers cette limite commune.
	\end{enumerate}
\end{solution}

\end{document}
